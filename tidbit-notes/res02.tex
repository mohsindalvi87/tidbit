


\section{Reviews}

%\begin{landscape}
%	After this point, everything is displayed in landscape format.\\
%\pagebreak
\begin{longtable}[c]{|p{0.25\textwidth}|p{0.13\textwidth}|p{0.30\textwidth}|p{0.2\textwidth}|} 
%\hline\hline \multicolumn{4}{| c |}{\textit{Begin of Notes}}\\ 
%\endfirsthead
\hline 	\textbf{ID, Title, Author, Journal} 	&  \textbf{Research areas, Tools}  &  \textbf{Objectives, Methodology, Discussion}  &  \textbf{Conclusions, My inference, Gaps} \\ \endhead
%\hline \endfoot
\hline\multicolumn{4}{| c |}{\textit{End of Notes}}\\ \hline\hline \endlastfoot

% ---------------------------------------------------------
\hline  \multiline { \item\textbf{authyear}    \item\textit{paper title}    \item firstname1 surname1 and firstname2 surname2    \item journal } 
&  \multiline{ \item keyword1    \item keyword2 }
&  \multiline{ \item objec    \item method    \item disc } 
&  \multiline {\item concl    \item gap } \\

% ---------------------------------------------------------
\hline  \multiline { \item\textbf{aydin2006}    \item\textit{Quaternion Based Inverse Kinematics for Industrial Robot Manipulators with Euler Wrist}    \item Yavuz Ayd{\i}n and Serdar Kucuk    \item IEEE 3rd International Conference on Mechatronics (ICM 2006) } 
&  \multiline{ \item wrist IK    \item quaternion-vector pair for 6-dof pose }
&  \multiline{ \item Double quaternion instead of dual quaternion    \item Approximate translation magnitue $d$ as rotation by angle $\psi = \dfrac{d}{R}$ about normalised vector $d$. } 
&  \multiline {\item Results obtained by using double quaternions are coordinate frame invariant. } \\

% ---------------------------------------------------------
\hline  \multiline { \item\textbf{authyear}    \item\textit{paper title}    \item firstname1 surname1 and firstname2 surname2    \item journal } 
&  \multiline{ \item keyword1    \item keyword2 }
&  \multiline{ \item objec    \item method    \item disc } 
&  \multiline {\item concl    \item gap } \\

% ---------------------------------------------------------
\hline  \multiline { \item\textbf{ge1998} \item\textit{Double quaternions for motion interpolation} \item Ge, Q J and Varshney, Amitabh and Menon, Jai P and Chang, Chu-Fei \item --- } 
&  \multiline{ \item dual quaternions }
&  \multiline{ \item Double quaternion instead of dual quaternion \item Approximate translation magnitue $d$ as rotation by angle $\psi = \dfrac{d}{R}$ about normalised vector $d$. } 
&  \multiline {\item Results obtained by using double quaternions are coordinate frame invariant. } \\

% ---------------------------------------------------------
\hline  \multiline { \item\textbf{lee2012a} \cite{lee2012a}    \item\textit{Estimation of {A}ttitude and {E}xternal {A}cceleration {U}sing {I}nertial {S}ensor {M}easurement {D}uring {V}arious {D}ynamic {C}onditions}    \item Jung Keun Lee and Edward J. Park and Stephen N. Robinovitch    \item IEEE Transactions on Instrumentation and Measurement } 
&  \multiline{ \item IMU  \item  attitude estimation  \item Kalman filter }
&  \multiline{ \item objec    \item method    \item disc } 
&  \multiline {\item concl    \item gap } \\

% ---------------------------------------------------------
\hline  \multiline { \item\textbf{authyear}    \item\textit{paper title}    \item firstname1 surname1 and firstname2 surname2    \item journal } 
&  \multiline{ \item keyword1    \item keyword2 }
&  \multiline{ \item objec    \item method    \item disc } 
&  \multiline {\item concl    \item gap } \\

% ---------------------------------------------------------
\hline  \multiline { \item\textbf{oland2018} \cite{oland2018a}    \item\textit{Quaternion-based Control of Fixed-Wing UAVs Using Logarithmic Mapping}    \item Espen Oland    \item 9th International Conference on Mechanical and Aerospace Engineering, IEEE } 
&  \multiline{ \item DQ log map    \item UAV stability analysis }
&  \multiline{ \item objec    \item method    \item disc } 
&  \multiline {\item concl    \item gap } \\

% ---------------------------------------------------------
\hline  \multiline { \item\textbf{oliveira2015} \cite{oliveira2015a} \item\textit{A new method of applying dfferential kinematics through dual quaternions} \item Andre Schneider de Oliveira and Edson Roberto De Pieri and Ubirajara Franco Moreno \item Robotica, Cambridge } 
&  \multiline{ \item DQ diff. kinematics, DQ Jacobian }
&  \multiline{ \item objec    \item method    \item disc } 
&  \multiline {\item concl    \item gap } \\

% ---------------------------------------------------------
\hline  \multiline { \item\textbf{authyear}    \item\textit{paper title}    \item firstname1 surname1 and firstname2 surname2    \item journal } 
&  \multiline{ \item keyword1    \item keyword2 }
&  \multiline{ \item objec    \item method    \item disc } 
&  \multiline {\item concl    \item gap } \\

% ---------------------------------------------------------
\hline \multiline{\item\textbf{thomas} \item\textit{Approaching Dual Quaternions from Matrix Algebra} \item Frederico Thomas \item --- }
&\multiline{\item double quaternion derivation}
&  \multiline{ \item objec    \item method    \item disc } 
&  \multiline {\item concl    \item gap } \\

% ---------------------------------------------------------
\hline \multiline{\item\textbf{wang2012a}\cite{wang2012a} \item\textit{The geometric structure of unit quaternion with application in kinematic control} \item Xiangke Wang and Dapeng Han and Changbin Yu and Zhiqiang Zheng \item Journal of Mathematical Analysis and Applications, Elsevier }
&\multiline{\item DQ geom. struc., DQ log mapping, kinematic control}
&  \multiline{ \item objec    \item method    \item disc } 
&  \multiline {\item concl    \item gap } \\

% ---------------------------------------------------------
\hline  \multiline { \item\textbf{authyear}    \item\textit{paper title}    \item firstname1 surname1 and firstname2 surname2    \item journal } 
&  \multiline{ \item keyword1    \item keyword2 }
&  \multiline{ \item objec    \item method    \item disc } 
&  \multiline {\item concl    \item gap } \\

% ---------------------------------------------------------
	
\end{longtable}
%\end{landscape}
%After this point, everything is displayed in portrait format.\\

