\documentclass[runningheads, 11pt]{article}

\usepackage[latin1]{inputenc}
\usepackage[T1]{fontenc}
\usepackage{amsmath,amssymb,
	amsthm} % amssymb loads amsfonts
\usepackage{bm} % bm for math italic-bold
\usepackage{mathptmx}  % [charter]{mathdesign} {lmodern} {fourier} {pxfonts} {mathptmx} {newtxtext,newtxmath} {mathpazo} {txfonts} {cmbright} [math]{iwona} [light,condensed,math]{kurier} {arev} {nimbusserif} {crimson} {stix} {times} {bera} [sfdefault]{roboto}
%\usepackage{newtxtext}  \usepackage[libertine]{newtxmath}
%\usepackage[charter]{mathdesign} %[charter][utopia][garamond] put % before amsfonts,amssymb

\usepackage[margin=2cm]{geometry}
\usepackage{lscape,enumitem,microtype}
\tolerance=1 \emergencystretch=\maxdimen \hyphenpenalty=10000 \hbadness=10000
\setlength{\parindent}{1em} \setlength{\parskip}{1em}
\usepackage{indentfirst}  % \usepackage{parskip} to remove parindent

\usepackage{physics}
\usepackage{siunitx}
\sisetup{per-mode=symbol-or-fraction, bracket-unit-denominator=false}
\usepackage{booktabs,multirow,multicol,makecell,array,longtable}
\usepackage{graphicx}
\usepackage{caption}
\usepackage[caption=false]{subfig}%{subcaption}

%\usepackage{algorithm}
%\usepackage{algorithmic}
%\algsetup{linenosize=\small,linenodelimiter=.}
%\DeclareCaptionFormat{algor}{%
%	\hrulefill\par\offinterlineskip\vskip1pt%
%	\textbf{#1#2}#3\offinterlineskip\hrulefill}
%\DeclareCaptionStyle{algori}{singlelinecheck=off,format=algor,labelsep=space}
%\captionsetup[algorithm]{style=algori}
\newcommand{\INPUT}{\item[\textbf{Input:}]}
\newcommand{\OUTPUT}{\item[\textbf{Output:}]}
\usepackage[ruled,linesnumbered,vlined]{algorithm2e} % [ruled,vlined]

\usepackage{hyperref}
\usepackage[nameinlink]{cleveref} %[noabbrev,nameinlink]

\usepackage{tikz}
%%from https://texample.net/tikz/examples/observer-estimator/
%	\usetikzlibrary{decorations.pathmorphing} % for snake lines
%	\usetikzlibrary{matrix} % for block alignment
%	\usetikzlibrary{arrows} % for arrow heads
%	\usetikzlibrary{shapes} %
%	\usetikzlibrary{calc} % for manimulation of coordinates
%	\tikzstyle{block} = [draw,rectangle,thick,minimum height=2em,minimum width=2em]
%	\tikzstyle{sum} = [draw,circle,inner sep=0mm,minimum size=2mm]
%	\tikzstyle{connector} = [->,thick]
%	\tikzstyle{line} = [thick]
%	\tikzstyle{branch} = [circle,inner sep=0pt,minimum size=1mm,fill=black,draw=black]
%	\tikzstyle{guide} = []
%	\tikzstyle{snakeline} = [connector, decorate, decoration={pre length=0.2cm,
%		post length=0.2cm, snake, amplitude=.4mm,
%		segment length=2mm},thick, magenta, ->]
%	\renewcommand{\vec}[1]{\ensuremath{\boldsymbol{#1}}} % bold vectors
%	\def \myneq {\skew{-2}\not =} % \neq alone skews the dash
%%from https://texample.net/tikz/examples/inertial-navigation-system/
%	\newcommand{\mx}[1]{\mathbf{\bm{#1}}} % Matrix command
%	\newcommand{\vc}[1]{\mathbf{\bm{#1}}} % Vector command
%	\tikzstyle{block} = [draw, fill=blue!20, rectangle, 
%	minimum height=3em, minimum width=6em]
%%from https://texample.net/tikz/examples/control-system-principles/
%	\usetikzlibrary{shapes,arrows}
%	\tikzstyle{sum} = [draw, fill=blue!20, circle, node distance=1cm]
%	\tikzstyle{input} = [coordinate]
%	\tikzstyle{output} = [coordinate]
%	\tikzstyle{pinstyle} = [pin edge={to-,thin,black}]
%%from https://texample.net/tikz/examples/noise-shaper/
%\usepackage{textcomp}


%\usetikzlibrary{dsp,fit}
%\usepackage{natbib}
%\setcitestyle{authoryear,square,citesep={,}}

\usepackage{littsumm}





\newcommand{\diag}{\text{diag}}
\newcommand{\tpose}[1][T]{^\text{#1}}
\newcommand{\bpose}[1]{_\text{#1}}
\newcommand{\Bmtrx}[1]{\begin{bmatrix}#1\end{bmatrix}}
\newcommand{\bmtrx}[1]{\left[ #1 \right]}
\newcommand{\vex}[1]{\left[ #1 \times \right]}
\newcommand{\multiline}[1]{\begin{itemize}[leftmargin=.75em,topsep=-1em,itemsep=0em]#1\end{itemize}}


\DeclareTextFontCommand{\mytexttt}{\ttfamily\hyphenchar\font=63-+\relax} % new command for hyphenation put =63 for line break after ?, =45 for line break after -
\newcommand\textvtt[1]{{\normalfont\fontfamily{cmvtt}\selectfont #1}} % variable typewriter font in computer modern









\begin{document}
	
\title{Research Notes}
\author{Mohsin Dalvi -- DT17MEC050}
\date{\today}
\maketitle

\LScitekey{{hol2008}}:\\
\LSauth{Jeroen Hol.
	\newblock {\em Pose Estimation and Calibration Algorithms for Vision and
		Inertial Sensors}.
	\newblock PhD thesis, Link\"{o}ping University, Sweden, 2008.}:\\
\LSjour{Jeroen Hol.
	\newblock {\em Pose Estimation and Calibration Algorithms for Vision and
		Inertial Sensors}.
	\newblock PhD thesis, Link\"{o}ping University, Sweden, 2008.}:\\
\LSyr{Jeroen Hol.
\newblock {\em Pose Estimation and Calibration Algorithms for Vision and
	Inertial Sensors}.
	\newblock PhD thesis, Link\"{o}ping University, Sweden, 2008.}:\\
\LSlink{Jeroen Hol.
	\newblock {\em Pose Estimation and Calibration Algorithms for Vision and
		Inertial Sensors}.
	\newblock PhD thesis, Link\"{o}ping University, Sweden, 2008.}:\\

	
Hello World! \cite{lentin2019}

%Dual quaternions (DQ), developed by Clifford, consist of a dual scalar and a dual vector as $ \underline{p} = \underline{p}_0 + \underline{\vec{p}} $. It is is rewritten to give $ \underline{p} = p_r + \epsilon p_d $, where 
%and $ p_r ,\, p_d \in \mathbb{H} $, and represented by an eight parameter tuple $( q_{p0}, q_{p1}, q_{p2}, q_{p3}, q_{d0}, q_{d1}, q_{d2}, q_{d3} ) $.
%
%\begin{equation}\label{eq:dqmult}
%\underline{p} \, \underline{q} = p_{r} q_{r} + \epsilon ( p_{r} q_{d} + p_{d} q_{r} ) =  \begin{bmatrix} H \left( p_r \right) & \vec{0} \\ H \left( p_d \right) & H \left( p_r \right) \end{bmatrix}  \begin{bmatrix} q_r \\ q_d \end{bmatrix}
%\end{equation} 
%where, \\
%$ H \left( p \right) = \begin{bmatrix} p_0 & \ -\vec{p}^T \\ \vec{p} & \ p_0 \bm{I} + \left[\vec{p}\right]_\times  \end{bmatrix} $ for vector $ \vec{p} = \left( p_1, p_2, p_3 \right) $, $ \bm{I} = \text{diag} \left( 1, 1, 1 \right) $  and $ \left[\vec{p}\right]_\times = \begin{bmatrix} 0 & -p_3 &\ p_2 \\ p_3 & 0 & -p_1 \\ -p_2 &\ p_1 & 0 \end{bmatrix}$.



%\begin{figure}[h]
%	\subfloat{ \includegraphics[width=0.4 \textwidth]{./imgs/trajectoryq4c.pdf}  }
%	\hfill
%	\subfloat{ \includegraphics[width=0.4 \textwidth]{./imgs/trajectoryq5c.pdf}  }
%	\caption{ Comparison of DQ elements ($q_{d0}$ and $q_{d1}$) traced by $J^T$ and DLS IK models \cite{fernandez2015}}  \label{fig:dqelements}
%\end{figure}



%\begin{table}[h] \centering
%	\caption{ Frame transformations carried out using \cite{lentin2015} } \label{tab:screwmotions}
%	\begin {tabular}[h]	{| c | c  c |}	 \hline 
%	%	\begin {tabular}[h]	{| c | p{0.35\textwidth} | p{0.35\textwidth} |}	 \hline		
%	\shortstack{Frame \\ transformation}   &   $\left\lbrace i-1 \right\rbrace$ to $\left\lbrace i' \right\rbrace$   &   $\left\lbrace i' \right\rbrace$ to $\left\lbrace i \right\rbrace$ \\ \hline
%	Rotation angle $ \theta $   &   $ \theta_i $   &   $ \alpha_i $ \\ \hline
%	Rotation axis $ \hat{\vec{u}} $   &   $ ( 0, 0, 1 ) $   &   $ ( 1, 0, 0 ) $ \\ \hline
%	\shortstack{Translation \\ direction $ \vec{t} $ }  &   $ ( 0, 0, d_i ) $   &   $ ( a_i, 0, 0 ) $ \\ \hline
%	$ \hat{\vec{u}} \cdot \vec{t} $   &   $ d_i $   &   $ a_i $ \\ \hline
%	$ \hat{\vec{u}} \times \vec{t} $   &   $ 0 $   &    $ 0 $ \\ \hline
%	\shortstack{Transformation \\ DQ} & 
%	\shortstack {  $ \underline{q}^Z = \left[ C_\theta, \, 0, \, 0, \, S_\theta \right] +  \qquad $   \\   $ \qquad \epsilon \left[ -  D S_\theta, \, 0, \, 0, \, D C_\theta \right] $   }   &
%	\shortstack {  $ \underline{q}^X = \left[ C_\alpha, S_\alpha, 0, 0 \right] +  \qquad $   \\   $ \qquad \epsilon \left[ - A S_\alpha,  A C_\alpha, 0, 0 \right] $  }  \\ \hline \end{tabular} 
%\end{table}



%\input{res01}





\section{Reviews}

%\begin{landscape}
%	After this point, everything is displayed in landscape format.\\
%\pagebreak
\begin{longtable}[c]{|p{0.25\textwidth}|p{0.13\textwidth}|p{0.30\textwidth}|p{0.2\textwidth}|} 
%\hline\hline \multicolumn{4}{| c |}{\textit{Begin of Notes}}\\ 
%\endfirsthead
\hline 	\textbf{ID, Title, Author, Journal} 	&  \textbf{Research areas, Tools}  &  \textbf{Objectives, Methodology, Discussion}  &  \textbf{Conclusions, My inference, Gaps} \\ \endhead
%\hline \endfoot
\hline\multicolumn{4}{| c |}{\textit{End of Notes}}\\ \hline\hline \endlastfoot

% ---------------------------------------------------------
\hline  \multiline { \item\textbf{authyear}    \item\textit{paper title}    \item firstname1 surname1 and firstname2 surname2    \item journal } 
&  \multiline{ \item keyword1    \item keyword2 }
&  \multiline{ \item objec    \item method    \item disc } 
&  \multiline {\item concl    \item gap } \\

% ---------------------------------------------------------
\hline  \multiline { \item\textbf{aydin2006}    \item\textit{Quaternion Based Inverse Kinematics for Industrial Robot Manipulators with Euler Wrist}    \item Yavuz Ayd{\i}n and Serdar Kucuk    \item IEEE 3rd International Conference on Mechatronics (ICM 2006) } 
&  \multiline{ \item wrist IK    \item quaternion-vector pair for 6-dof pose }
&  \multiline{ \item Double quaternion instead of dual quaternion    \item Approximate translation magnitue $d$ as rotation by angle $\psi = \dfrac{d}{R}$ about normalised vector $d$. } 
&  \multiline {\item Results obtained by using double quaternions are coordinate frame invariant. } \\

% ---------------------------------------------------------
\hline  \multiline { \item\textbf{authyear}    \item\textit{paper title}    \item firstname1 surname1 and firstname2 surname2    \item journal } 
&  \multiline{ \item keyword1    \item keyword2 }
&  \multiline{ \item objec    \item method    \item disc } 
&  \multiline {\item concl    \item gap } \\

% ---------------------------------------------------------
\hline  \multiline { \item\textbf{ge1998} \item\textit{Double quaternions for motion interpolation} \item Ge, Q J and Varshney, Amitabh and Menon, Jai P and Chang, Chu-Fei \item --- } 
&  \multiline{ \item dual quaternions }
&  \multiline{ \item Double quaternion instead of dual quaternion \item Approximate translation magnitue $d$ as rotation by angle $\psi = \dfrac{d}{R}$ about normalised vector $d$. } 
&  \multiline {\item Results obtained by using double quaternions are coordinate frame invariant. } \\

% ---------------------------------------------------------
\hline  \multiline { \item\textbf{laviola2003} \cite{laviola2003a}    \item\textit{A Comparison of Unscented and Extended Kalrnan Filtering for Estimating Quaternion Motion}   \item Joseph J. LaViola Jr.    \item Proceedings of the American Control Conference Denver, Colorado June 4-6,2003 } 
&  \multiline{ \item Quaternion    \item EKF \item UKF }
&  \multiline{ \item objec    \item method sampling rates 25, 80, 215 \si{Hz}   \item Ground truth obtained by passing data through zero phase shift filter to remove high-frequency noise. \item Avg over Monte Carlo runs taken and RMS error of rotation parameter $ \theta $ is calculated as $ \sqrt{ \tfrac{1}{2} \sum \limits ^{n-1} _{i=0} e^{2}_{i}} $ where $ e_{i} = \tfrac{2(180)}{\pi} \arccos( \text{Sc}(q_{i} (\hat{q}_{i})^{-1} ) ) $   \item disc } 
&  \multiline {\item concl    \item gap } \\

% ---------------------------------------------------------
\hline  \multiline { \item\textbf{lee2012a} \cite{lee2012a}    \item\textit{Estimation of {A}ttitude and {E}xternal {A}cceleration {U}sing {I}nertial {S}ensor {M}easurement {D}uring {V}arious {D}ynamic {C}onditions}    \item Jung Keun Lee and Edward J. Park and Stephen N. Robinovitch    \item IEEE Transactions on Instrumentation and Measurement } 
&  \multiline{ \item IMU  \item  attitude estimation  \item Kalman filter }
&  \multiline{ \item objec    \item method    \item disc } 
&  \multiline {\item concl    \item gap } \\

% ---------------------------------------------------------
\hline  \multiline { \item\textbf{authyear}    \item\textit{paper title}    \item firstname1 surname1 and firstname2 surname2    \item journal } 
&  \multiline{ \item keyword1    \item keyword2 }
&  \multiline{ \item objec    \item method    \item disc } 
&  \multiline {\item concl    \item gap } \\

% ---------------------------------------------------------
\hline  \multiline { \item\textbf{oland2018} \cite{oland2018a}    \item\textit{Quaternion-based Control of Fixed-Wing UAVs Using Logarithmic Mapping}    \item Espen Oland    \item 9th International Conference on Mechanical and Aerospace Engineering, IEEE } 
&  \multiline{ \item DQ log map    \item UAV stability analysis }
&  \multiline{ \item objec    \item method    \item disc } 
&  \multiline {\item concl    \item gap } \\

% ---------------------------------------------------------
\hline  \multiline { \item\textbf{oliveira2015} \cite{oliveira2015a} \item\textit{A new method of applying dfferential kinematics through dual quaternions} \item Andre Schneider de Oliveira and Edson Roberto De Pieri and Ubirajara Franco Moreno \item Robotica, Cambridge } 
&  \multiline{ \item DQ diff. kinematics, DQ Jacobian }
&  \multiline{ \item objec    \item method    \item disc } 
&  \multiline {\item concl    \item gap } \\

% ---------------------------------------------------------
\hline  \multiline { \item\textbf{authyear}    \item\textit{paper title}    \item firstname1 surname1 and firstname2 surname2    \item journal } 
&  \multiline{ \item keyword1    \item keyword2 }
&  \multiline{ \item objec    \item method    \item disc } 
&  \multiline {\item concl    \item gap } \\

% ---------------------------------------------------------
\hline \multiline{\item\textbf{thomas} \item\textit{Approaching Dual Quaternions from Matrix Algebra} \item Frederico Thomas \item --- }
&\multiline{\item double quaternion derivation}
&  \multiline{ \item objec    \item method    \item disc } 
&  \multiline {\item concl    \item gap } \\

% ---------------------------------------------------------
\hline \multiline{\item\textbf{wang2012a}\cite{wang2012a} \item\textit{The geometric structure of unit quaternion with application in kinematic control} \item Xiangke Wang and Dapeng Han and Changbin Yu and Zhiqiang Zheng \item Journal of Mathematical Analysis and Applications, Elsevier }
&\multiline{\item DQ geom. struc., DQ log mapping, kinematic control}
&  \multiline{ \item objec    \item method    \item disc } 
&  \multiline {\item concl    \item gap } \\

% ---------------------------------------------------------
\hline  \multiline { \item\textbf{authyear}    \item\textit{paper title}    \item firstname1 surname1 and firstname2 surname2    \item journal } 
&  \multiline{ \item keyword1    \item keyword2 }
&  \multiline{ \item objec    \item method    \item disc } 
&  \multiline {\item concl    \item gap } \\

% ---------------------------------------------------------
	
\end{longtable}
%\end{landscape}
%After this point, everything is displayed in portrait format.\\




%\input{res03}





\bibliographystyle{unsrt}
\bibliography{../../biblio/md_biblio} % mybibliography

\end{document}