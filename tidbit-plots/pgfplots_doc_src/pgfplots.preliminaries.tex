
\chapter[About PGFPlots: Preliminaries]{About {\normalfont\PGFPlots{}}: Preliminaries}

This section contains information about upgrades, the team, the installation
(in case you need to do it manually) and troubleshooting. You may skip it
completely except for the upgrade remarks.

\PGFPlots{} is built completely on \Tikz{}/\PGF{}. Knowledge of \Tikz{} will
simplify the work with \PGFPlots{}, although it is not required.

\section{Components}

\PGFPlots{} comes with two components:
%
\begin{enumerate}
    \item the plotting component (which you are currently reading) and
    \item the \PGFPlotstable{} component which simplifies number formatting
        and postprocessing of numerical tables. It comes as a separate
        package and has its own manual
        \href{file:pgfplotstable.pdf}{pgfplotstable.pdf}.
\end{enumerate}


\section{Upgrade remarks}

This release provides a lot of improvements which can be found in all detail in
\texttt{ChangeLog} for interested readers. However, some attention is useful
with respect to the following changes.

One thing which is common to \PGFPlots{} is the key |compat|: it is strongly
suggested to always write it into your |.tex| files. While this key imposes
some work to end-users, it also solves a common requirement: it ensures that
your |.tex| files always result in the same output, even if you install a new
version of \PGFPlots{}. On the other hand, it allows us as maintainers to solve
software defects and introduce changes in behavior, assuming that these changes
only affect documents with a decent |compat|ibility level. The precise impact
of the |compat| key, its choices and implications are described in the
following sections.


\subsection{New Optional Features}

\PGFPlots{} has been written with backwards compatibility in mind: old \TeX{}
files should compile without modifications and without changes in the
appearance. However, new features occasionally lead to a different behavior. In
such a case, \PGFPlots{} will deactivate the new feature.\footnote{In case of
broken backwards compatibility, we apologize -- and ask you to submit a bug
report. We will take care of it.}

Any new features or bugfixes which cause backwards compatibility problems need
to be activated \emph{manually} and \emph{explicitly}. In order to do so, you
should use
%
\input pgfplots.preliminaries.compatcurrent.tmp

%
\noindent in your preamble. This will configure the compatibility layer.

You should have at least |compat=1.3|. The suggested value is printed to the
|.log| file after running \TeX{}.

Here is a list of changes which are inactive unless one uses a suitable
|compat| level:
%
\begin{enumerate}
    \item \PGFPlots{} 1.18 and 1.17 have no additional constraints and is the same as
        1.16 with respect to compatibility levels.
    \item \PGFPlots{} 1.16 has no additional constraints and is the same as
        1.15 with respect to compatibility levels.
    \item \PGFPlots{} 1.15 activates |3d log sampling| and repairs issues
        with |clip limits| for bar plots.
    \item \PGFPlots{} 1.14 changes the way nonuniform colormaps are handled
        by the system and activates advanced |colormap| operations (see
        |of colormap|).
    \item \PGFPlots{} 1.13 repairs axis labels in polar axis and ensures that
        the color chosen by |shader=flat| is independent of |z buffer| and
        |mesh/ordering|. Furthermore, it enables |stack negative=separate|
        for all stacked bar plots. Older compatibility levels are present to
        keep workarounds by end-users.
    \item \PGFPlots{} 1.12 activates |lua backend| and defines
        |boxplot/estimator=Excel|.
    \item \PGFPlots{} 1.11 changes the |axis cs|: it is now the default
        coordinate system. If you write |\draw (1,2) -- (2,2);| \PGFPlots{}
        will automatically treat it as
        |\draw (axis cs:1,2) -- (axis cs:2,2);|.
    \item \PGFPlots{} 1.10 has no differences to 1.9 with respect to
        compatibility.
    \item \PGFPlots{} 1.9 comes with a preset to combine |ybar stacked| and
        |nodes near coords|. Furthermore, it suppresses empty increments in
        stacked bar plots. In order to activate the new preset, you have to
        use |compat=1.9| or higher.
    \item \PGFPlots{} 1.8 comes with a new revision for alignment of label
        and tick scale label alignment. Furthermore, it improves the bounding
        box for |hide axis|. This revision is enabled with |compat=1.8| or
        higher.

        The configuration |compat=1.8| is \emph{necessary} to repair
        |axis lines=center| in three-dimensional axes.
    \item \PGFPlots{} 1.7 added new options for bar widths defined in terms
        of axis units. These are enabled with |compat=1.7| or higher.
    \item \PGFPlots{} 1.6 added new options for more accurate scaling and
        more scaling options for |\addplot3 graphics|. These are enabled with
        |compat=1.6| or higher.
    \item \PGFPlots{} 1.5.1 interprets circle and ellipse radii as
        \PGFPlots{} coordinates (older versions used \pgfname{} unit vectors
        which have no direct relation to \PGFPlots). In other words: starting
        with version 1.5.1, it is possible to write |\draw circle[radius=5]|
        inside of an axis. This requires |\pgfplotsset{compat=1.5.1}| or
        higher.

        Without this compatibility setting, circles and ellipses use
        low-level canvas units of \pgfname{} as in earlier versions.
    \item \PGFPlots{} 1.5 uses |log origin=0| as default (which influences
        logarithmic bar plots or stacked logarithmic plots). Older versions
        keep |log origin=infty|. This requires |\pgfplotsset{compat=1.5}| or
        higher.
    \item \PGFPlots{} 1.4 has fixed several smaller bugs which might produce
        differences of about $1$--$2\text{pt}$ compared to earlier releases.
        This requires |\pgfplotsset{compat=1.4}| or higher.
    \item \PGFPlots{} 1.3 comes with user interface improvements. The
        technical distinction between ``behavior options'' and ``style
        options'' of older versions is no longer necessary (although still
        fully supported).

        This is always activated.
    \item \PGFPlots{} 1.3 has a new feature which allows to \emph{move axis
        labels tight to tick labels} automatically. This is strongly
        recommended. It requires |\pgfplotsset{compat=1.3}| or higher.

        Since this affects the spacing, it is not enabled be default.
    \item \PGFPlots{} 1.3 supports reversed axes. It is no longer necessary
        to use workarounds with negative units. \pgfkeys{/pdflinks/search key
        prefixes in/.add={/pgfplots/,}{}}

        Take a look at the |x dir=reverse| key.

        Existing workarounds will still function properly. Use
        |\pgfplotsset{compat=1.3}| or higher together with |x dir=reverse| to
        switch to the new version.
\end{enumerate}


\subsection{Old Features Which May Need Attention}

\begin{enumerate}
    \item The |scatter/classes| feature produces proper legends as of version
        1.3. This may change the appearance of existing legends of plots with
        |scatter/classes|.
    \item Starting with \PGFPlots{} 1.1, |\tikzstyle| should \emph{no longer
        be used} to set \PGFPlots{} options.

        Although |\tikzstyle| is still supported for some older \PGFPlots{}
        options, you should replace any occurrence of |\tikzstyle| with
        |\pgfplotsset{|\meta{style name}|/.style={|\meta{key-value-list}|}}|
        or the associated |/.append style| variant. See
        Section~\ref{sec:styles} for more detail.
\end{enumerate}
%
I apologize for any inconvenience caused by these changes.

\begin{pgfplotskey}{%
    compat=\mchoice{
        1.18,
        1.17,
        1.16,
        1.15,
        1.14,
        1.13,
        1.12,
        1.11,
        1.10,
        1.9,
        1.8,
        1.7,
        1.6,
        1.5.1,
        1.5,
        1.4,
        1.3,
        pre 1.3,
        default
    } (initially default)}
    The preamble configuration
    %
\input pgfplots.preliminaries.compatcurrent.tmp
    %
    allows to choose between backwards compatibility and most recent features.
    This key is designed to be the first encountered \PGFPlots{} key in a
    document as it prepares global options.

    Occasionally, you might want to use different versions in the same
    document. Then, provide
    %
\begin{codeexample}[code only]
\begin{figure}
    \pgfplotsset{compat=1.4}
    ...
    \caption{...}
\end{figure}
\end{codeexample}
    %
    \noindent in order to restrict the compatibility setting to the actual
    context (in this case, the |figure| environment).

    See the output of your |.log| file to get a suggested value for |compat|.

    Use |\pgfplotsset{compat=default}| to restore the factory settings.

    Although typically unnecessary, it is also possible to activate only
    selected changes and keep compatibility to older versions in general:
    %
    \begin{pgfplotskeylist}{%
        compat/path replacement=\meta{version},
        compat/labels=\meta{version},
        compat/scaling=\meta{version},
        compat/scale mode=\meta{version},
        compat/empty line=\meta{version},
        compat/plot3graphics=\meta{version},
        compat/bar nodes=\meta{version},
        compat/BB=\meta{version},
        compat/bar width by units=\meta{version},
        compat/pgfpoint substitution=\meta{version},
        compat/general=\meta{version}
    }
        Let us assume that we have a document with |\pgfplotsset{compat=1.3}|
        and you want to keep it this way.

        In addition, you realized that version 1.5.1 supports circles and
        ellipses. Then, use
\begin{codeexample}[]
% preamble:
\pgfplotsset{
    compat=1.3,
    compat/path replacement=1.5.1,
}
\begin{tikzpicture}
\begin{axis}[
    extra x ticks={-2,2},
    extra y ticks={-2,2},
    extra tick style={grid=major}]
    \addplot {x};
    \draw (axis cs:0,0) circle[radius=2];
\end{axis}
\end{tikzpicture}
\end{codeexample}

        All of these keys accept the possible values of the |compat| key.

        The |compat/path replacement| key controls how radii of circles and
        ellipses are interpreted.

        The |compat/labels| key controls how axis labels are aligned: either
        uses adjacent to ticks or with an absolute offset. As of |1.8|, it also
        enables an entirely new revision of the axis label styles. In most
        cases, you will see no difference -- but it repairs |axis lines=center|
        in three-dimensional axes.

        The |compat/scaling| key controls some bugfixes introduced in version
        1.4 and 1.6: they might introduce slight scaling differences in order
        to improve the accuracy.

        The |compat/plot3graphics| controls new features for
        |\addplot3 graphics|.

        The |compat/scale mode| allows to enable/disable the warning ``The
        content of your 3d axis has CHANGED compared to previous versions''
        because the |axis equal| and |unit vector ratio| features where broken
        for all versions before~1.6 and have been fixed in~1.6.

        The |compat/empty line| allows to write empty lines into input files in
        order to generate a jump. This requires |compat=1.4| or newer. See
        |empty line| for details.

        The |compat/BB| changes to bounding box to be tight even in case of
        |hide axis|.

        The |compat/bar width by units| allows to express |bar width=1| (i.e.\@
        in terms of axis units).

        The |compat/bar nodes| activates presets for |ybar stacked| and
        |nodes near coords|. In addition, it enables |stacked ignores zero| for
        stacked bar plots.

        The |compat/general| key controls |log origin|, |lua backend|,
        |enable tick line clipping|, and |boxplot/estimator|.

        The |compat/pgfpoint substitution| key determines if |axis cs| is the
        default coordinate system (as of 1.11).

        The detailed effects can be seen on the beginning of this section.
    \end{pgfplotskeylist}

    The value \meta{version} can be |default|, a version number, and |newest|.
    The value |default| is the same as |pre 1.3| (up to insignificant changes).
    The use of |newest| is strongly \emph{discouraged}: it might cause changes
    in your document, depending on the current version of \PGFPlots{}. Please
    inspect your |.log| file to see suggestions for the best possible version.
\end{pgfplotskey}


\section{The Team}

\PGFPlots{} has been written mainly by Christian Feuersänger with many
improvements of Pascal Wolkotte and Nick Papior Andersen as a spare time
project. We hope it is useful and provides valuable plots.

If you are interested in writing something but don't know how, consider reading
the auxiliary manual
\href{file:TeX-programming-notes.pdf}{TeX-programming-notes.pdf} which comes
with \PGFPlots{}. It is far from complete, but maybe it is a good starting
point (at least for more literature).


\section{Acknowledgements}

I thank God for all hours of enjoyed programming. I thank Pascal Wolkotte and
Nick Papior Andersen for their programming efforts and contributions as part of
the development team. My thanks go to Francesco Poli for implementing the |lua| algorithm for |contour lua|.
I thank Jürnjakob Dugge for his contribution of
|hist/density|, matlab scripts for \verbpdfref{\addplot3} |graphics|, excellent
user forum help and helpful bug reports. I thank Stefan Tibus, who contributed
the |\addplot shell| feature. I thank Tom Cashman for the contribution of the
|reverse legend| feature. Special thanks go to Stefan Pinnow for his continuous
efforts to test \PGFPlots{}, to discuss requirements, to request features and
bug fixes which lead to numerous quality improvements, and to adapt and
integrate the |colorbrewer| library. Furthermore, I thank Prof.~Schweitzer for
many fruitful discussions and his initial encouragement to start such a
package. Thanks to Dr.~Meine for his ideas and suggestions. Special thanks go
to Markus Böhning for proof-reading all the manuals of \PGF{}, \PGFPlots{}, and
\PGFPlotstable{}. Thanks to Vincent A. Traag for bringing |colorbrewer| colors
to \PGFPlots{}. Thanks as well to the many international contributors who
provided feature requests or identified bugs or simply improvements of the
manual!

Last but not least, I thank Till Tantau and Mark Wibrow for their excellent
graphics (and more) package \PGF{} and \Tikz{}, which is the base of
\PGFPlots{}.


% main=manual.tex

\section{Installation and Prerequisites}

\subsection{Licensing}

This program is free software: you can redistribute it and/or modify it under
the terms of the GNU General Public License as published by the Free Software
Foundation, either version 3 of the License, or (at your option) any later
version.

This program is distributed in the hope that it will be useful, but WITHOUT ANY
WARRANTY; without even the implied warranty of MERCHANTABILITY or FITNESS FOR A
PARTICULAR PURPOSE.  See the GNU General Public License for more details.

A copy of the GNU General Public License can be found in the package file
%
\begin{verbatim}
doc/latex/pgfplots/gpl-3.0.txt
\end{verbatim}
%
You may also visit~\url{http://www.gnu.org/licenses}.


\subsection{Prerequisites}

\PGFPlots{} requires \PGF{}. You should generally use the most recent stable
version of \PGF{}. \PGFPlots{} is used with
%
\input pgfplots.preliminaries.compatcurrent.tmp
%
\noindent in your preamble (see Section~\ref{sec:tex:dialects} for information
about how to use it with Con\TeX{}t and plain \TeX{}).

The |compat=|\meta{yourversion} entry should be added to activate new features,
see the documentation of the |compat| key for more details.


%\subsection{Installation}

There are several ways how to teach \TeX{} where to find the files. Choose the
option which fits your needs best.


\subsection{Installation in Windows}

Windows users often use MiK\TeX{} which downloads the latest stable package
versions automatically. You do not need to install anything manually here.

If you want to install or more recent version of \PGFPlots{} than the one
shipped with MiK\TeX{}, you can proceed as follows. MiK\TeX{} provides a
feature to install packages locally in its own \TeX{} Directory Structure
(TDS). The basic idea is to unzip \PGFPlots{} in a directory of your choice and
configure the MiK\TeX{} Package Manager to use this specific directory with
higher priority than its default paths. If you want to do this, start the
MiK\TeX{} Settings using ``Start $\gg$ Programs $\gg$ MiK\TeX{} $\gg$
Settings''. There, use the ``Roots'' menu section. It contains the MiK\TeX{}
Package directory as initial configuration. Use ``Add'' to select the directory
in which the unzipped \PGFPlots{} tree resides. Then, move the newly added path
to the list's top using the ``Up'' button. Then press ``Ok''. For MiK\TeX{}
2.8, you may need to uncheck the ``Show MiK\TeX{}-maintained root directories''
button to see the newly installed path.

MiK\TeX{} complains if the provided directory is not TDS conform (see
Section~\ref{pgfplots:tds} for details), so you can't provide a wrong directory
here. This method does also work for other packages, but some packages may need
some directory restructuring before MiK\TeX{} accepts them.


\subsection{Installation of Linux Packages}

Typically, \PGFPlots{} can be installed using the \TeX{} package manager. A
common distribution is \TeX{}Live. In this case you can write
%
\begin{codeexample}[code only]
sudo tlmgr install pgfplots
\end{codeexample}
%
\noindent in order to install \PGFPlots{}.


\subsection{Installation in Any Directory -- the \texttt{TEXINPUTS} Variable}

You can simply install \PGFPlots{} anywhere on your hard drive, for example
into
%
\begin{verbatim}
/foo/bar/pgfplots.
\end{verbatim}
%
Then, you set the \texttt{TEXINPUTS} variable to
%
\begin{verbatim}
TEXINPUTS=/foo/bar/pgfplots/tex//:
\end{verbatim}
%
The trailing~`\texttt{:}' tells \TeX{} to check the default search paths after
\lstinline!/foo/bar/pgfplots!. The double slash~`\texttt{//}' tells \TeX{} to
search all subdirectories.

If the \texttt{TEXINPUTS} variable already contains something, you can append
the line above to the existing \texttt{TEXINPUTS} content.

Furthermore, you should set |TEXDOCS| as well,
%
\begin{verbatim}
TEXDOCS=/foo/bar/pgfplots/doc//:
\end{verbatim}
%
so that the \TeX{} documentation system finds the files |pgfplots.pdf| and
|pgfplotstable.pdf| (on some systems, it is then enough to use
|texdoc pgfplots|).

Starting with \PGFPlots{} 1.12, you may also need to adopt \texttt{LUAINPUTS}:
%
\begin{verbatim}
LUAINPUTS=/foo/bar/pgfplots//:
\end{verbatim}
%
should usually do the job.

Please refer to your operating systems manual for how to set environment
variables.


\subsection{Installation Into a Local TDS Compliant \texttt{texmf}-Directory}
\label{pgfplots:tds}

\PGFPlots{} comes in a ``\TeX{} Directory Structure'' (TDS) conforming
directory structure, so you can simply unpack the files into a directory which
is searched by \TeX{} automatically. Such directories are |~/texmf| on Linux
systems, for example.

Copy \PGFPlots{} to a local \texttt{texmf} directory like \lstinline!~/texmf!.
You need at least the \PGFPlots{} directories |tex/generic/pgfplots| and
|tex/latex/pgfplots|. Then, run \lstinline!texhash! (or some equivalent
path-updating command specific to your \TeX{} distribution).

The TDS consists of several sub directories which are searched separately,
depending on what has been requested: the sub directories
|doc/latex/|\meta{package} are used for (\LaTeX{}) documentation, the
sub-directories |doc/generic/|\meta{package} for documentation which apply to
\LaTeX{} and other \TeX{} dialects (like plain \TeX{} and Con\TeX{}t which have
their own, respective sub-directories) as well.

Similarly, the |tex/latex/|\meta{package} sub-directories are searched whenever
\LaTeX{} packages are requested. The |tex/generic/|\meta{package}
sub-directories are searched for packages which work for \LaTeX{} \emph{and}
other \TeX{} dialects.

Do not forget to run \lstinline!texhash!.

\subsection{Installation If Everything Else Fails\ldots}

If \TeX{} still doesn't find your files, you can copy all \lstinline!.sty! and
all |.code.tex| files (perhaps all |.def| files as well) into your current
project's working directory. In fact, you need everything which is in the
|tex/latex/pgfplots| and |tex/generic/pgfplots| sub directories.

Please refer to \url{http://www.ctan.org/installationadvice/} for more
information about package installation.


\section{Troubleshooting -- Error Messages}

This section discusses some problems which may occur when using \PGFPlots{}.
Some of the error messages are shown in the index, take a look at the end of
this manual (under ``Errors'').


\subsection{Problems with available Dimen registers}

To avoid problems with the many required \TeX{} registers for \PGF{} and
\PGFPlots{}, you may want to include
%
\begin{verbatim}
\usepackage{etex}
\end{verbatim}
%
as first package. This avoids problems with ``no room for a new
dimen''\index{Error Messages!No room for a new dimen} in most cases. It should
work with any modern installation of \TeX{} (it activates the
$\varepsilon$-\TeX{} extensions).


\subsection{Dimension Too Large Errors}

The core mathematical engine of \PGF{} relies on \TeX{} registers to perform
fast arithmetics. To compute $50+299$, it actually computes |50pt+299pt| and
strips the |pt| suffix of the result. Since \TeX{} registers can only contain
numbers up to $\pm 16384$, overflow error messages like ``Dimension too large''
occur if the result leaves the allowed range. Normally, this should never
happen -- \PGFPlots{} uses a floating point unit with data range $\pm 10^{324}$
and performs all mappings automatically. However, there are some cases where
this fails. Some of these cases are:
%
\begin{enumerate}
    \item The axis range (for example, for $x$) becomes \emph{relatively}
        small. It's no matter if you have absolutely small ranges like
        $[10^{-17},10^{-16}]$. But if you have an axis range like
        $[1.99999999,2]$, where a lot of significant digits are necessary,
        this may be problematic.

        I guess I can't help here: you may need to prepare the data somehow
        before \PGFPlots{} processes it.
    \item This may happen as well if you only view a very small portion of
        the data range.

        This happens, for example, if your input data ranges from $x\in
        [0,10^6]$, and you say |xmax=10|.

        Consider using the |restrict x to domain*=|\meta{min}|:|\meta{max}
        key in such a case, where the \meta{min} and \meta{max} should be
        (say) four times of your axis limits (see
        page~\pageref{key:restrict:x:to:domain} for details).
    \item The |axis equal| key will be confused if $x$ and $y$ have a very
        different scale.
    \item You may have found a bug -- please contact the developers.
\end{enumerate}


\subsection{Restrictions for DVI Viewers and \texttt{dvipdfm}}
\label{sec:drivers}

\PGF{} is compatible with
%
\begin{itemize}
    \item \lstinline!latex!/\lstinline!dvips!,
    \item \lstinline!latex!/\lstinline!dvipdfm!,
    \item \lstinline!pdflatex!,
    \item $\vdots$
\end{itemize}
%
However, there are some restrictions: I don't know any DVI viewer which is
capable of viewing the output of \PGF{} (and therefor \PGFPlots{} as well).
After all, DVI has never been designed to draw something different than text
and horizontal/vertical lines. You will need to view the postscript file or the
PDF file.

Then, the DVI/PDF combination doesn't support all types of shadings (for
example, the |shader=interp| is only available for |dvips|, |pdftex|,
|dvipdfmx|, and |xetex| drivers).

Furthermore, \PGF{} needs to know a \emph{driver} so that the DVI file can be
converted to the desired output. Depending on your system, you need the
following options:
%
\begin{itemize}
    \item \lstinline!latex!/\lstinline!dvips! does not need anything special
        because \lstinline!dvips! is the default driver if you invoke
        \lstinline!latex!.
    \item \lstinline!pdflatex! will also work directly because
        \lstinline!pdflatex! will be detected automatically.
    \item \lstinline!lualatex! will also be detected automatically.
    \item \lstinline!latex!/\lstinline!dvipdfm! requires to use
        %
\begin{verbatim}
\def\pgfsysdriver{pgfsys-dvipdfm.def}
%\def\pgfsysdriver{pgfsys-pdftex.def}
%\def\pgfsysdriver{pgfsys-dvips.def}
%\def\pgfsysdriver{pgfsys-dvipdfmx.def}
%\def\pgfsysdriver{pgfsys-xetex.def}
%\def\pgfsysdriver{pgfsys-luatex.def}
\usepackage{pgfplots}.
\end{verbatim}
        %
        The uncommented commands could be used to set other drivers
        explicitly.
\end{itemize}
%
Please read the corresponding sections in~\cite[Sections~7.2.1 and 7.2.2]{tikz}
if you have further questions. These sections also contain limitations of
particular drivers.

The choice which won't produce any problems at all is |pdflatex|.


\subsection{Problems with \TeX's Memory Capacities}

\PGFPlots{} can handle small up to medium sized plots. However, \TeX{} has
never been designed for data plots -- you will eventually face the problem of
small memory capacities. See Section~\ref{sec:pgfplots:optimization} for how to
enlarge them.


\subsection{Problems with Language Settings and Active Characters}

Both \PGF{} and \PGFPlots{} use a lot of active characters -- which may lead to
incompatibilities with other packages which define active characters.
Compatibility is better than in earlier versions, but may still be an issue.
The manual compiles with the |babel| package for English and French, the
|german| package does also work. If you experience any trouble, let me know.
Sometimes it may work to disable active characters temporarily (|babel|
provides such a command).


\subsection{Other Problems}

Please read the mailing list at
\url{http://sourceforge.net/projects/pgfplots/support}. Perhaps someone has
also encountered your problem before, and maybe he came up with a solution.

Please write a note on the mailing list if you have a different problem. In
case it is necessary to contact the authors directly, consider the addresses
shown on the title page of this document.

